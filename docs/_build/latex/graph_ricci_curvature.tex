%% Generated by Sphinx.
\def\sphinxdocclass{report}
\documentclass[letterpaper,10pt,english]{sphinxmanual}
\ifdefined\pdfpxdimen
   \let\sphinxpxdimen\pdfpxdimen\else\newdimen\sphinxpxdimen
\fi \sphinxpxdimen=.75bp\relax
\ifdefined\pdfimageresolution
    \pdfimageresolution= \numexpr \dimexpr1in\relax/\sphinxpxdimen\relax
\fi
%% let collapsible pdf bookmarks panel have high depth per default
\PassOptionsToPackage{bookmarksdepth=5}{hyperref}

\PassOptionsToPackage{booktabs}{sphinx}
\PassOptionsToPackage{colorrows}{sphinx}

\PassOptionsToPackage{warn}{textcomp}
\usepackage[utf8]{inputenc}
\ifdefined\DeclareUnicodeCharacter
% support both utf8 and utf8x syntaxes
  \ifdefined\DeclareUnicodeCharacterAsOptional
    \def\sphinxDUC#1{\DeclareUnicodeCharacter{"#1}}
  \else
    \let\sphinxDUC\DeclareUnicodeCharacter
  \fi
  \sphinxDUC{00A0}{\nobreakspace}
  \sphinxDUC{2500}{\sphinxunichar{2500}}
  \sphinxDUC{2502}{\sphinxunichar{2502}}
  \sphinxDUC{2514}{\sphinxunichar{2514}}
  \sphinxDUC{251C}{\sphinxunichar{251C}}
  \sphinxDUC{2572}{\textbackslash}
\fi
\usepackage{cmap}
\usepackage[T1]{fontenc}
\usepackage{amsmath,amssymb,amstext}
\usepackage{babel}



\usepackage{tgtermes}
\usepackage{tgheros}
\renewcommand{\ttdefault}{txtt}



\usepackage[Bjarne]{fncychap}
\usepackage{sphinx}

\fvset{fontsize=auto}
\usepackage{geometry}


% Include hyperref last.
\usepackage{hyperref}
% Fix anchor placement for figures with captions.
\usepackage{hypcap}% it must be loaded after hyperref.
% Set up styles of URL: it should be placed after hyperref.
\urlstyle{same}

\addto\captionsenglish{\renewcommand{\contentsname}{Contents:}}

\usepackage{sphinxmessages}
\setcounter{tocdepth}{1}



\title{graph\_ricci\_curvature}
\date{Apr 25, 2024}
\release{0.1.0}
\author{Brian Andrews}
\newcommand{\sphinxlogo}{\vbox{}}
\renewcommand{\releasename}{Release}
\makeindex
\begin{document}

\ifdefined\shorthandoff
  \ifnum\catcode`\=\string=\active\shorthandoff{=}\fi
  \ifnum\catcode`\"=\active\shorthandoff{"}\fi
\fi

\pagestyle{empty}
\sphinxmaketitle
\pagestyle{plain}
\sphinxtableofcontents
\pagestyle{normal}
\phantomsection\label{\detokenize{index::doc}}


\sphinxstepscope


\chapter{graph\_ricci\_curvature}
\label{\detokenize{modules:graph-ricci-curvature}}\label{\detokenize{modules::doc}}
\sphinxstepscope


\section{graph\_ricci\_curvature package}
\label{\detokenize{graph_ricci_curvature:graph-ricci-curvature-package}}\label{\detokenize{graph_ricci_curvature::doc}}

\subsection{Submodules}
\label{\detokenize{graph_ricci_curvature:submodules}}

\subsection{graph\_ricci\_curvature.graph\_metric module}
\label{\detokenize{graph_ricci_curvature:module-graph_ricci_curvature.graph_metric}}\label{\detokenize{graph_ricci_curvature:graph-ricci-curvature-graph-metric-module}}\index{module@\spxentry{module}!graph\_ricci\_curvature.graph\_metric@\spxentry{graph\_ricci\_curvature.graph\_metric}}\index{graph\_ricci\_curvature.graph\_metric@\spxentry{graph\_ricci\_curvature.graph\_metric}!module@\spxentry{module}}\index{GraphMetric (class in graph\_ricci\_curvature.graph\_metric)@\spxentry{GraphMetric}\spxextra{class in graph\_ricci\_curvature.graph\_metric}}

\begin{fulllineitems}
\phantomsection\label{\detokenize{graph_ricci_curvature:graph_ricci_curvature.graph_metric.GraphMetric}}
\pysigstartsignatures
\pysiglinewithargsret{\sphinxbfcode{\sphinxupquote{class\DUrole{w}{ }}}\sphinxcode{\sphinxupquote{graph\_ricci\_curvature.graph\_metric.}}\sphinxbfcode{\sphinxupquote{GraphMetric}}}{\sphinxparam{\DUrole{n}{G}\DUrole{p}{:}\DUrole{w}{ }\DUrole{n}{Graph}}\sphinxparamcomma \sphinxparam{\DUrole{n}{weight\_key}}}{}
\pysigstopsignatures
\sphinxAtStartPar
Bases: \sphinxcode{\sphinxupquote{ABC}}

\sphinxAtStartPar
Parent class for classes calculating properties of a graph


\subsubsection{Parameters}
\label{\detokenize{graph_ricci_curvature:parameters}}\begin{description}
\sphinxlineitem{G}{[}networkx graph{]}
\sphinxAtStartPar
Input graph

\sphinxlineitem{weight\_key}{[}str{]}
\sphinxAtStartPar
key to specify edge weights in networkx dictionary

\end{description}

\end{fulllineitems}



\subsection{graph\_ricci\_curvature.ollivier\_ricci\_curvature module}
\label{\detokenize{graph_ricci_curvature:module-graph_ricci_curvature.ollivier_ricci_curvature}}\label{\detokenize{graph_ricci_curvature:graph-ricci-curvature-ollivier-ricci-curvature-module}}\index{module@\spxentry{module}!graph\_ricci\_curvature.ollivier\_ricci\_curvature@\spxentry{graph\_ricci\_curvature.ollivier\_ricci\_curvature}}\index{graph\_ricci\_curvature.ollivier\_ricci\_curvature@\spxentry{graph\_ricci\_curvature.ollivier\_ricci\_curvature}!module@\spxentry{module}}\begin{description}
\sphinxlineitem{References:}\begin{itemize}
\item {} 
\sphinxAtStartPar
Ollivier, Y. 2009. “Ricci curvature of Markov chains on metric spaces”. Journal of Functional Analysis, 256(3), 810\sphinxhyphen{}864.

\item {} 
\sphinxAtStartPar
Sandhu et al. 2015. “Graph Curvature for Differentiating Cancer Networks”. Scientific Reports. DOi: 10.1038/srep12323

\end{itemize}

\end{description}
\index{OllivierRicciCurvature (class in graph\_ricci\_curvature.ollivier\_ricci\_curvature)@\spxentry{OllivierRicciCurvature}\spxextra{class in graph\_ricci\_curvature.ollivier\_ricci\_curvature}}

\begin{fulllineitems}
\phantomsection\label{\detokenize{graph_ricci_curvature:graph_ricci_curvature.ollivier_ricci_curvature.OllivierRicciCurvature}}
\pysigstartsignatures
\pysiglinewithargsret{\sphinxbfcode{\sphinxupquote{class\DUrole{w}{ }}}\sphinxcode{\sphinxupquote{graph\_ricci\_curvature.ollivier\_ricci\_curvature.}}\sphinxbfcode{\sphinxupquote{OllivierRicciCurvature}}}{\sphinxparam{\DUrole{n}{G}\DUrole{p}{:}\DUrole{w}{ }\DUrole{n}{Graph}}\sphinxparamcomma \sphinxparam{\DUrole{n}{weight\_key}\DUrole{o}{=}\DUrole{default_value}{\textquotesingle{}weight\textquotesingle{}}}}{}
\pysigstopsignatures
\sphinxAtStartPar
Bases: {\hyperref[\detokenize{graph_ricci_curvature:graph_ricci_curvature.ricci_curvature.RicciCurvature}]{\sphinxcrossref{\sphinxcode{\sphinxupquote{RicciCurvature}}}}}

\sphinxAtStartPar
Class for calculating Ollivier Ricci Curvature


\subsubsection{Parameters}
\label{\detokenize{graph_ricci_curvature:id1}}\begin{description}
\sphinxlineitem{G}{[}networkx graph{]}
\sphinxAtStartPar
Input graph

\sphinxlineitem{weight\_key}{[}str{]}
\sphinxAtStartPar
key to specify edge weights in networkx dictionary.

\end{description}
\index{calculate\_edge\_curvature() (graph\_ricci\_curvature.ollivier\_ricci\_curvature.OllivierRicciCurvature method)@\spxentry{calculate\_edge\_curvature()}\spxextra{graph\_ricci\_curvature.ollivier\_ricci\_curvature.OllivierRicciCurvature method}}

\begin{fulllineitems}
\phantomsection\label{\detokenize{graph_ricci_curvature:graph_ricci_curvature.ollivier_ricci_curvature.OllivierRicciCurvature.calculate_edge_curvature}}
\pysigstartsignatures
\pysiglinewithargsret{\sphinxbfcode{\sphinxupquote{calculate\_edge\_curvature}}}{\sphinxparam{\DUrole{n}{source\_node}}\sphinxparamcomma \sphinxparam{\DUrole{n}{target\_node}}\sphinxparamcomma \sphinxparam{\DUrole{n}{alpha}\DUrole{o}{=}\DUrole{default_value}{0.5}}}{}
\pysigstopsignatures
\sphinxAtStartPar
Calculate value of Ricci Curvature tensor associated with an edge
between a source and target node defined as

\sphinxAtStartPar
1 \sphinxhyphen{} ( Wasserstein 1 Distance / Edge Weight )


\paragraph{Parameters}
\label{\detokenize{graph_ricci_curvature:id2}}\begin{description}
\sphinxlineitem{source\_node}{[}int or tuple{]}
\sphinxAtStartPar
index of source\_node in graph self.G

\sphinxlineitem{target\_node}{[}int or tuple{]}
\sphinxAtStartPar
index of target node in graph self.G

\sphinxlineitem{alpha}{[}float{]}
\sphinxAtStartPar
hyperparameter (0 \textless{}= alpha \textless{}=1) determining how much mass to move
from node

\end{description}


\paragraph{Returns}
\label{\detokenize{graph_ricci_curvature:returns}}\begin{description}
\sphinxlineitem{curvature}{[}float{]}
\sphinxAtStartPar
value of curvature tensor

\end{description}

\end{fulllineitems}

\index{calculate\_ricci\_curvature() (graph\_ricci\_curvature.ollivier\_ricci\_curvature.OllivierRicciCurvature method)@\spxentry{calculate\_ricci\_curvature()}\spxextra{graph\_ricci\_curvature.ollivier\_ricci\_curvature.OllivierRicciCurvature method}}

\begin{fulllineitems}
\phantomsection\label{\detokenize{graph_ricci_curvature:graph_ricci_curvature.ollivier_ricci_curvature.OllivierRicciCurvature.calculate_ricci_curvature}}
\pysigstartsignatures
\pysiglinewithargsret{\sphinxbfcode{\sphinxupquote{calculate\_ricci\_curvature}}}{\sphinxparam{\DUrole{n}{alpha}\DUrole{o}{=}\DUrole{default_value}{0.5}}\sphinxparamcomma \sphinxparam{\DUrole{n}{norm}\DUrole{o}{=}\DUrole{default_value}{True}}}{}
\pysigstopsignatures
\sphinxAtStartPar
Calculate nonzero values of Ricci curvature tensor for all edges in
graph self.G


\paragraph{Parameters}
\label{\detokenize{graph_ricci_curvature:id3}}\begin{description}
\sphinxlineitem{alpha}{[}float{]}
\sphinxAtStartPar
hyperparameter (0 \textless{}= alpha \textless{}=1) determining how much mass to move
from node

\sphinxlineitem{norm}{[}bool{]}
\sphinxAtStartPar
if True, normalize nodal scalar curvature

\end{description}


\paragraph{Returns}
\label{\detokenize{graph_ricci_curvature:id4}}\begin{description}
\sphinxlineitem{self.G}{[}networkx graph{]}
\sphinxAtStartPar
Returns graph with ricci\_curvature as node and edge attributes

\end{description}

\end{fulllineitems}


\end{fulllineitems}



\subsection{graph\_ricci\_curvature.ricci\_curvature module}
\label{\detokenize{graph_ricci_curvature:module-graph_ricci_curvature.ricci_curvature}}\label{\detokenize{graph_ricci_curvature:graph-ricci-curvature-ricci-curvature-module}}\index{module@\spxentry{module}!graph\_ricci\_curvature.ricci\_curvature@\spxentry{graph\_ricci\_curvature.ricci\_curvature}}\index{graph\_ricci\_curvature.ricci\_curvature@\spxentry{graph\_ricci\_curvature.ricci\_curvature}!module@\spxentry{module}}\index{RicciCurvature (class in graph\_ricci\_curvature.ricci\_curvature)@\spxentry{RicciCurvature}\spxextra{class in graph\_ricci\_curvature.ricci\_curvature}}

\begin{fulllineitems}
\phantomsection\label{\detokenize{graph_ricci_curvature:graph_ricci_curvature.ricci_curvature.RicciCurvature}}
\pysigstartsignatures
\pysiglinewithargsret{\sphinxbfcode{\sphinxupquote{class\DUrole{w}{ }}}\sphinxcode{\sphinxupquote{graph\_ricci\_curvature.ricci\_curvature.}}\sphinxbfcode{\sphinxupquote{RicciCurvature}}}{\sphinxparam{\DUrole{n}{G}\DUrole{p}{:}\DUrole{w}{ }\DUrole{n}{Graph}}\sphinxparamcomma \sphinxparam{\DUrole{n}{weight\_key}}}{}
\pysigstopsignatures
\sphinxAtStartPar
Bases: {\hyperref[\detokenize{graph_ricci_curvature:graph_ricci_curvature.graph_metric.GraphMetric}]{\sphinxcrossref{\sphinxcode{\sphinxupquote{GraphMetric}}}}}

\sphinxAtStartPar
Class for storing information about the Ricci Curvature Tensor


\subsubsection{Parameters}
\label{\detokenize{graph_ricci_curvature:id5}}\begin{description}
\sphinxlineitem{G}{[}networkx graph{]}
\sphinxAtStartPar
Input graph

\sphinxlineitem{weight\_key}{[}str{]}
\sphinxAtStartPar
key to specify edge weights in networkx dictionary. Default = weight

\end{description}

\end{fulllineitems}



\subsection{Module contents}
\label{\detokenize{graph_ricci_curvature:module-graph_ricci_curvature}}\label{\detokenize{graph_ricci_curvature:module-contents}}\index{module@\spxentry{module}!graph\_ricci\_curvature@\spxentry{graph\_ricci\_curvature}}\index{graph\_ricci\_curvature@\spxentry{graph\_ricci\_curvature}!module@\spxentry{module}}

\chapter{Indices and tables}
\label{\detokenize{index:indices-and-tables}}\begin{itemize}
\item {} 
\sphinxAtStartPar
\DUrole{xref,std,std-ref}{genindex}

\item {} 
\sphinxAtStartPar
\DUrole{xref,std,std-ref}{modindex}

\item {} 
\sphinxAtStartPar
\DUrole{xref,std,std-ref}{search}

\end{itemize}


\renewcommand{\indexname}{Python Module Index}
\begin{sphinxtheindex}
\let\bigletter\sphinxstyleindexlettergroup
\bigletter{g}
\item\relax\sphinxstyleindexentry{graph\_ricci\_curvature}\sphinxstyleindexpageref{graph_ricci_curvature:\detokenize{module-graph_ricci_curvature}}
\item\relax\sphinxstyleindexentry{graph\_ricci\_curvature.graph\_metric}\sphinxstyleindexpageref{graph_ricci_curvature:\detokenize{module-graph_ricci_curvature.graph_metric}}
\item\relax\sphinxstyleindexentry{graph\_ricci\_curvature.ollivier\_ricci\_curvature}\sphinxstyleindexpageref{graph_ricci_curvature:\detokenize{module-graph_ricci_curvature.ollivier_ricci_curvature}}
\item\relax\sphinxstyleindexentry{graph\_ricci\_curvature.ricci\_curvature}\sphinxstyleindexpageref{graph_ricci_curvature:\detokenize{module-graph_ricci_curvature.ricci_curvature}}
\end{sphinxtheindex}

\renewcommand{\indexname}{Index}
\printindex
\end{document}