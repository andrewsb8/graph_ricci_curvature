%% Generated by Sphinx.
\def\sphinxdocclass{report}
\documentclass[letterpaper,10pt,english]{sphinxmanual}
\ifdefined\pdfpxdimen
   \let\sphinxpxdimen\pdfpxdimen\else\newdimen\sphinxpxdimen
\fi \sphinxpxdimen=.75bp\relax
\ifdefined\pdfimageresolution
    \pdfimageresolution= \numexpr \dimexpr1in\relax/\sphinxpxdimen\relax
\fi
%% let collapsible pdf bookmarks panel have high depth per default
\PassOptionsToPackage{bookmarksdepth=5}{hyperref}

\PassOptionsToPackage{booktabs}{sphinx}
\PassOptionsToPackage{colorrows}{sphinx}

\PassOptionsToPackage{warn}{textcomp}
\usepackage[utf8]{inputenc}
\ifdefined\DeclareUnicodeCharacter
% support both utf8 and utf8x syntaxes
  \ifdefined\DeclareUnicodeCharacterAsOptional
    \def\sphinxDUC#1{\DeclareUnicodeCharacter{"#1}}
  \else
    \let\sphinxDUC\DeclareUnicodeCharacter
  \fi
  \sphinxDUC{00A0}{\nobreakspace}
  \sphinxDUC{2500}{\sphinxunichar{2500}}
  \sphinxDUC{2502}{\sphinxunichar{2502}}
  \sphinxDUC{2514}{\sphinxunichar{2514}}
  \sphinxDUC{251C}{\sphinxunichar{251C}}
  \sphinxDUC{2572}{\textbackslash}
\fi
\usepackage{cmap}
\usepackage[T1]{fontenc}
\usepackage{amsmath,amssymb,amstext}
\usepackage{babel}



\usepackage{tgtermes}
\usepackage{tgheros}
\renewcommand{\ttdefault}{txtt}



\usepackage[Bjarne]{fncychap}
\usepackage{sphinx}

\fvset{fontsize=auto}
\usepackage{geometry}


% Include hyperref last.
\usepackage{hyperref}
% Fix anchor placement for figures with captions.
\usepackage{hypcap}% it must be loaded after hyperref.
% Set up styles of URL: it should be placed after hyperref.
\urlstyle{same}

\addto\captionsenglish{\renewcommand{\contentsname}{Contents:}}

\usepackage{sphinxmessages}
\setcounter{tocdepth}{1}



\title{graph\_ricci\_curvature}
\date{Jul 01, 2024}
\release{1.0.0}
\author{Brian Andrews}
\newcommand{\sphinxlogo}{\vbox{}}
\renewcommand{\releasename}{Release}
\makeindex
\begin{document}

\ifdefined\shorthandoff
  \ifnum\catcode`\=\string=\active\shorthandoff{=}\fi
  \ifnum\catcode`\"=\active\shorthandoff{"}\fi
\fi

\pagestyle{empty}
\sphinxmaketitle
\pagestyle{plain}
\sphinxtableofcontents
\pagestyle{normal}
\phantomsection\label{\detokenize{index::doc}}



\chapter{graph\_ricci\_curvature}
\label{\detokenize{index:graph-ricci-curvature}}
\sphinxAtStartPar
Calculate the Ricci curvature tensor for a networkx graph. Both Ollivier {[}1{]} and Forman {[}3{]} discretizations of Ricci curvature are implemented (see {[}4{]} for comparison of methods).


\section{Installation}
\label{\detokenize{index:installation}}

\subsection{From Source}
\label{\detokenize{index:from-source}}\begin{itemize}
\item {} 
\sphinxAtStartPar
Clone the repository and \sphinxcode{\sphinxupquote{cd}} into the top level directory

\item {} 
\sphinxAtStartPar
Install python’s \sphinxcode{\sphinxupquote{build}}: \sphinxcode{\sphinxupquote{python \sphinxhyphen{}m pip install build}}

\item {} 
\sphinxAtStartPar
Build the project: \sphinxcode{\sphinxupquote{python \sphinxhyphen{}m build}}

\item {} 
\sphinxAtStartPar
Install the project with pip: \sphinxcode{\sphinxupquote{python \sphinxhyphen{}m pip install dist/{[}file name{]}.whl}}

\end{itemize}

\sphinxAtStartPar
For testing the installation, you need \sphinxcode{\sphinxupquote{pytest}}. Run \sphinxcode{\sphinxupquote{pytest}} in the top level directory to run the full test suite.


\subsection{Download the .whl from Releases}
\label{\detokenize{index:download-the-whl-from-releases}}
\sphinxAtStartPar
After download, install the wheel via pip: \sphinxcode{\sphinxupquote{python \sphinxhyphen{}m pip install {[}file name{]}.whl}}


\subsection{From PyPi}
\label{\detokenize{index:from-pypi}}
\sphinxAtStartPar
Not done yet


\section{Usage}
\label{\detokenize{index:usage}}
\sphinxAtStartPar
After installation:

\begin{sphinxVerbatim}[commandchars=\\\{\}]
\PYG{k+kn}{from} \PYG{n+nn}{graph\PYGZus{}ricci\PYGZus{}curvature}\PYG{n+nn}{.}\PYG{n+nn}{ollivier\PYGZus{}ricci\PYGZus{}curvature} \PYG{k+kn}{import} \PYG{n}{OllivierRicciCurvature}
\PYG{k+kn}{import} \PYG{n+nn}{networkx} \PYG{k}{as} \PYG{n+nn}{nx}

\PYG{c+c1}{\PYGZsh{}setting up a simple graph}
\PYG{n}{G} \PYG{o}{=} \PYG{n}{nx}\PYG{o}{.}\PYG{n}{Graph}\PYG{p}{(}\PYG{p}{)}
\PYG{n}{G}\PYG{o}{.}\PYG{n}{add\PYGZus{}nodes\PYGZus{}from}\PYG{p}{(}\PYG{p}{[}\PYG{l+m+mi}{1}\PYG{p}{,} \PYG{l+m+mi}{2}\PYG{p}{,} \PYG{l+m+mi}{3}\PYG{p}{]}\PYG{p}{)}
\PYG{n}{G}\PYG{o}{.}\PYG{n}{add\PYGZus{}edges\PYGZus{}from}\PYG{p}{(}\PYG{p}{[}\PYG{p}{(}\PYG{l+m+mi}{1}\PYG{p}{,} \PYG{l+m+mi}{2}\PYG{p}{)}\PYG{p}{,} \PYG{p}{(}\PYG{l+m+mi}{1}\PYG{p}{,} \PYG{l+m+mi}{3}\PYG{p}{)}\PYG{p}{]}\PYG{p}{)}

\PYG{c+c1}{\PYGZsh{}create an object and calculate the values of the tensor and its contractions}
\PYG{n}{g} \PYG{o}{=} \PYG{n}{OllivierRicciCurvature}\PYG{p}{(}\PYG{n}{G}\PYG{p}{)}
\PYG{n}{g}\PYG{o}{.}\PYG{n}{calculate\PYGZus{}ricci\PYGZus{}curvature}\PYG{p}{(}\PYG{p}{)}

\PYG{c+c1}{\PYGZsh{}print results of calculation}
\PYG{n+nb}{print}\PYG{p}{(}\PYG{n+nb}{list}\PYG{p}{(}\PYG{n}{g}\PYG{o}{.}\PYG{n}{G}\PYG{o}{.}\PYG{n}{edges}\PYG{o}{.}\PYG{n}{data}\PYG{p}{(}\PYG{p}{)}\PYG{p}{)}\PYG{p}{)}
\PYG{n+nb}{print}\PYG{p}{(}\PYG{n+nb}{list}\PYG{p}{(}\PYG{n}{g}\PYG{o}{.}\PYG{n}{G}\PYG{o}{.}\PYG{n}{nodes}\PYG{o}{.}\PYG{n}{data}\PYG{p}{(}\PYG{p}{)}\PYG{p}{)}\PYG{p}{)}
\PYG{n+nb}{print}\PYG{p}{(}\PYG{n}{g}\PYG{o}{.}\PYG{n}{G}\PYG{o}{.}\PYG{n}{graph}\PYG{p}{[}\PYG{l+s+s2}{\PYGZdq{}}\PYG{l+s+s2}{graph\PYGZus{}ricci\PYGZus{}curvature}\PYG{l+s+s2}{\PYGZdq{}}\PYG{p}{]}\PYG{p}{,} \PYG{n}{g}\PYG{o}{.}\PYG{n}{G}\PYG{o}{.}\PYG{n}{graph}\PYG{p}{[}\PYG{l+s+s2}{\PYGZdq{}}\PYG{l+s+s2}{norm\PYGZus{}graph\PYGZus{}ricci\PYGZus{}curvature}\PYG{l+s+s2}{\PYGZdq{}}\PYG{p}{]}\PYG{p}{)}
\end{sphinxVerbatim}

\sphinxAtStartPar
Output:

\begin{sphinxVerbatim}[commandchars=\\\{\}]
\PYG{c+c1}{\PYGZsh{}edge curvature data}
\PYG{p}{[}
\PYG{p}{(}\PYG{l+m+mi}{1}\PYG{p}{,} \PYG{l+m+mi}{2}\PYG{p}{,} \PYG{p}{\PYGZob{}}\PYG{l+s+s2}{\PYGZdq{}}\PYG{l+s+s2}{weight}\PYG{l+s+s2}{\PYGZdq{}}\PYG{p}{:} \PYG{l+m+mf}{1.0}\PYG{p}{,} \PYG{l+s+s2}{\PYGZdq{}}\PYG{l+s+s2}{ricci\PYGZus{}curvature}\PYG{l+s+s2}{\PYGZdq{}}\PYG{p}{:} \PYG{l+m+mf}{0.5}\PYG{p}{\PYGZcb{}}\PYG{p}{)}\PYG{p}{,}
\PYG{p}{(}\PYG{l+m+mi}{1}\PYG{p}{,} \PYG{l+m+mi}{3}\PYG{p}{,} \PYG{p}{\PYGZob{}}\PYG{l+s+s2}{\PYGZdq{}}\PYG{l+s+s2}{weight}\PYG{l+s+s2}{\PYGZdq{}}\PYG{p}{:} \PYG{l+m+mf}{1.0}\PYG{p}{,} \PYG{l+s+s2}{\PYGZdq{}}\PYG{l+s+s2}{ricci\PYGZus{}curvature}\PYG{l+s+s2}{\PYGZdq{}}\PYG{p}{:} \PYG{l+m+mf}{0.5}\PYG{p}{\PYGZcb{}}\PYG{p}{)}\PYG{p}{,}
\PYG{p}{]}

\PYG{c+c1}{\PYGZsh{}node curvature data}
\PYG{p}{[}
\PYG{p}{(}\PYG{l+m+mi}{1}\PYG{p}{,} \PYG{p}{\PYGZob{}}\PYG{l+s+s2}{\PYGZdq{}}\PYG{l+s+s2}{weight}\PYG{l+s+s2}{\PYGZdq{}}\PYG{p}{:} \PYG{l+m+mf}{1.0}\PYG{p}{,} \PYG{l+s+s2}{\PYGZdq{}}\PYG{l+s+s2}{ricci\PYGZus{}curvature}\PYG{l+s+s2}{\PYGZdq{}}\PYG{p}{:} \PYG{l+m+mf}{0.5}\PYG{p}{\PYGZcb{}}\PYG{p}{)}\PYG{p}{,}
\PYG{p}{(}\PYG{l+m+mi}{2}\PYG{p}{,} \PYG{p}{\PYGZob{}}\PYG{l+s+s2}{\PYGZdq{}}\PYG{l+s+s2}{weight}\PYG{l+s+s2}{\PYGZdq{}}\PYG{p}{:} \PYG{l+m+mf}{1.0}\PYG{p}{,} \PYG{l+s+s2}{\PYGZdq{}}\PYG{l+s+s2}{ricci\PYGZus{}curvature}\PYG{l+s+s2}{\PYGZdq{}}\PYG{p}{:} \PYG{l+m+mf}{0.5}\PYG{p}{\PYGZcb{}}\PYG{p}{)}\PYG{p}{,}
\PYG{p}{(}\PYG{l+m+mi}{3}\PYG{p}{,} \PYG{p}{\PYGZob{}}\PYG{l+s+s2}{\PYGZdq{}}\PYG{l+s+s2}{weight}\PYG{l+s+s2}{\PYGZdq{}}\PYG{p}{:} \PYG{l+m+mf}{1.0}\PYG{p}{,} \PYG{l+s+s2}{\PYGZdq{}}\PYG{l+s+s2}{ricci\PYGZus{}curvature}\PYG{l+s+s2}{\PYGZdq{}}\PYG{p}{:} \PYG{l+m+mf}{0.5}\PYG{p}{\PYGZcb{}}\PYG{p}{)}
\PYG{p}{]}

\PYG{c+c1}{\PYGZsh{}graph curvature data}
\PYG{l+m+mf}{1.5} \PYG{l+m+mf}{0.5}
\end{sphinxVerbatim}


\section{Manual}
\label{\detokenize{index:manual}}
\sphinxAtStartPar
You can see the manual \sphinxhref{https://github.com/andrewsb8/graph\_ricci\_curvature/blob/docs/docs/\_build/latex/graph\_ricci\_curvature.pdf}{here} which is in \sphinxcode{\sphinxupquote{docs/\_build/latex}}. Or, after installation, can run the following with python

\begin{sphinxVerbatim}[commandchars=\\\{\}]
\PYG{k+kn}{import} \PYG{n+nn}{graph\PYGZus{}ricci\PYGZus{}curvature} \PYG{k}{as} \PYG{n+nn}{grc}
\PYG{n}{grc}\PYG{o}{.}\PYG{n}{\PYGZus{}\PYGZus{}manual\PYGZus{}\PYGZus{}}
\end{sphinxVerbatim}

\sphinxAtStartPar
to obtain the link to the pdf in the github repository.


\chapter{References}
\label{\detokenize{index:references}}\begin{itemize}
\item {} 
\sphinxAtStartPar
{[}1{]} Ollivier, Y. 2009. “Ricci curvature of Markov chains on metric spaces”. Journal of Functional Analysis, 256(3), 810\sphinxhyphen{}864. DOI: https://doi.org/10.1016/j.jfa.2008.11.001, arXiv: https://arxiv.org/abs/math/0701886

\item {} 
\sphinxAtStartPar
{[}2{]} Sandhu et al. 2015. “Graph Curvature for Differentiating Cancer Networks”. Scientific Reports. DOi: 10.1038/srep12323. DOI: https://doi.org/10.1038/srep12323.

\item {} 
\sphinxAtStartPar
{[}3{]} R P Sreejith et al. “Forman curvature for complex networks” J. Stat. Mech. (2016) 063206. DOI: 10.1088/1742\sphinxhyphen{}5468/2016/06/063206. arXiv: https://arxiv.org/pdf/1603.00386.

\item {} 
\sphinxAtStartPar
{[}4{]} Samal et al. “Comparative analysis of two discretizations of Ricci curvature for complex networks”. Nature Scientific Reports, 2018. https://www.nature.com/articles/s41598\sphinxhyphen{}018\sphinxhyphen{}27001\sphinxhyphen{}3.

\end{itemize}

\sphinxstepscope


\chapter{graph\_ricci\_curvature}
\label{\detokenize{modules:graph-ricci-curvature}}\label{\detokenize{modules::doc}}
\sphinxstepscope


\section{graph\_ricci\_curvature package}
\label{\detokenize{graph_ricci_curvature:graph-ricci-curvature-package}}\label{\detokenize{graph_ricci_curvature::doc}}

\subsection{Submodules}
\label{\detokenize{graph_ricci_curvature:submodules}}

\subsection{graph\_ricci\_curvature.forman\_ricci\_curvature module}
\label{\detokenize{graph_ricci_curvature:module-graph_ricci_curvature.forman_ricci_curvature}}\label{\detokenize{graph_ricci_curvature:graph-ricci-curvature-forman-ricci-curvature-module}}\index{module@\spxentry{module}!graph\_ricci\_curvature.forman\_ricci\_curvature@\spxentry{graph\_ricci\_curvature.forman\_ricci\_curvature}}\index{graph\_ricci\_curvature.forman\_ricci\_curvature@\spxentry{graph\_ricci\_curvature.forman\_ricci\_curvature}!module@\spxentry{module}}\begin{description}
\sphinxlineitem{References:}\begin{itemize}
\item {} 
\sphinxAtStartPar
{[}1{]} R P Sreejith et al J. Stat. Mech. (2016) 063206. DOI: 10.1088/1742\sphinxhyphen{}5468/2016/06/063206. arXiv: \sphinxurl{https://arxiv.org/pdf/1603.00386}.

\end{itemize}

\end{description}
\index{FormanRicciCurvature (class in graph\_ricci\_curvature.forman\_ricci\_curvature)@\spxentry{FormanRicciCurvature}\spxextra{class in graph\_ricci\_curvature.forman\_ricci\_curvature}}

\begin{fulllineitems}
\phantomsection\label{\detokenize{graph_ricci_curvature:graph_ricci_curvature.forman_ricci_curvature.FormanRicciCurvature}}
\pysigstartsignatures
\pysiglinewithargsret{\sphinxbfcode{\sphinxupquote{class\DUrole{w}{ }}}\sphinxcode{\sphinxupquote{graph\_ricci\_curvature.forman\_ricci\_curvature.}}\sphinxbfcode{\sphinxupquote{FormanRicciCurvature}}}{\sphinxparam{\DUrole{n}{G}\DUrole{p}{:}\DUrole{w}{ }\DUrole{n}{Graph}}\sphinxparamcomma \sphinxparam{\DUrole{n}{edge\_weight\_key}\DUrole{o}{=}\DUrole{default_value}{\textquotesingle{}weight\textquotesingle{}}}\sphinxparamcomma \sphinxparam{\DUrole{n}{node\_weight\_key}\DUrole{o}{=}\DUrole{default_value}{\textquotesingle{}weight\textquotesingle{}}}}{}
\pysigstopsignatures
\sphinxAtStartPar
Bases: \sphinxcode{\sphinxupquote{\_RicciCurvature}}

\sphinxAtStartPar
Class for calculating Forman Ricci Curvature for a connected graph. Edge and
node weights are set to 1.0 unless values are specified by the user in the input
networkx graph object.


\subsubsection{Parameters}
\label{\detokenize{graph_ricci_curvature:parameters}}\begin{description}
\sphinxlineitem{G}{[}networkx graph{]}
\sphinxAtStartPar
Input graph

\sphinxlineitem{edge\_weight\_key}{[}str{]}
\sphinxAtStartPar
Key to specify edge weights in networkx graph. Default = weight.

\sphinxlineitem{node\_weight\_key}{[}str{]}
\sphinxAtStartPar
Key to specify node weights in networkx graph. Default = weight.

\end{description}
\index{calculate\_edge\_curvature() (graph\_ricci\_curvature.forman\_ricci\_curvature.FormanRicciCurvature method)@\spxentry{calculate\_edge\_curvature()}\spxextra{graph\_ricci\_curvature.forman\_ricci\_curvature.FormanRicciCurvature method}}

\begin{fulllineitems}
\phantomsection\label{\detokenize{graph_ricci_curvature:graph_ricci_curvature.forman_ricci_curvature.FormanRicciCurvature.calculate_edge_curvature}}
\pysigstartsignatures
\pysiglinewithargsret{\sphinxbfcode{\sphinxupquote{calculate\_edge\_curvature}}}{\sphinxparam{\DUrole{n}{source\_node}}\sphinxparamcomma \sphinxparam{\DUrole{n}{target\_node}}}{}
\pysigstopsignatures
\sphinxAtStartPar
Calculate value of Forman Ricci Curvature tensor associated with an edge
between a source and target node defined as in References.


\paragraph{Parameters}
\label{\detokenize{graph_ricci_curvature:id1}}\begin{description}
\sphinxlineitem{source\_node}{[}int or tuple{]}
\sphinxAtStartPar
index of source\_node in graph self.G

\sphinxlineitem{target\_node}{[}int or tuple{]}
\sphinxAtStartPar
index of target node in graph self.G

\end{description}

\end{fulllineitems}

\index{calculate\_ricci\_curvature() (graph\_ricci\_curvature.forman\_ricci\_curvature.FormanRicciCurvature method)@\spxentry{calculate\_ricci\_curvature()}\spxextra{graph\_ricci\_curvature.forman\_ricci\_curvature.FormanRicciCurvature method}}

\begin{fulllineitems}
\phantomsection\label{\detokenize{graph_ricci_curvature:graph_ricci_curvature.forman_ricci_curvature.FormanRicciCurvature.calculate_ricci_curvature}}
\pysigstartsignatures
\pysiglinewithargsret{\sphinxbfcode{\sphinxupquote{calculate\_ricci\_curvature}}}{\sphinxparam{\DUrole{n}{norm}\DUrole{o}{=}\DUrole{default_value}{True}}}{}
\pysigstopsignatures
\sphinxAtStartPar
Calculate nonzero values of Ricci curvature tensor for all edges in
graph self.G


\paragraph{Parameters}
\label{\detokenize{graph_ricci_curvature:id2}}\begin{description}
\sphinxlineitem{norm}{[}bool{]}
\sphinxAtStartPar
If True, normalize nodal scalar curvature.

\end{description}


\paragraph{Returns}
\label{\detokenize{graph_ricci_curvature:returns}}\begin{description}
\sphinxlineitem{self.G}{[}networkx graph{]}
\sphinxAtStartPar
Returns graph with ricci\_curvature as graph, node, and edge attributes

\end{description}

\end{fulllineitems}


\end{fulllineitems}



\subsection{graph\_ricci\_curvature.ollivier\_ricci\_curvature module}
\label{\detokenize{graph_ricci_curvature:module-graph_ricci_curvature.ollivier_ricci_curvature}}\label{\detokenize{graph_ricci_curvature:graph-ricci-curvature-ollivier-ricci-curvature-module}}\index{module@\spxentry{module}!graph\_ricci\_curvature.ollivier\_ricci\_curvature@\spxentry{graph\_ricci\_curvature.ollivier\_ricci\_curvature}}\index{graph\_ricci\_curvature.ollivier\_ricci\_curvature@\spxentry{graph\_ricci\_curvature.ollivier\_ricci\_curvature}!module@\spxentry{module}}\begin{description}
\sphinxlineitem{References:}\begin{itemize}
\item {} 
\sphinxAtStartPar
{[}1{]} Ollivier, Y. 2009. “Ricci curvature of Markov chains on metric spaces”. Journal of Functional Analysis, 256(3), 810\sphinxhyphen{}864. DOI: \sphinxurl{https://doi.org/10.1016/j.jfa.2008.11.001}, arXiv: \sphinxurl{https://arxiv.org/abs/math/0701886}

\item {} 
\sphinxAtStartPar
{[}2{]} Sandhu et al. 2015. “Graph Curvature for Differentiating Cancer Networks”. Scientific Reports. DOi: 10.1038/srep12323. DOI: \sphinxurl{https://doi.org/10.1038/srep12323}.

\end{itemize}

\end{description}
\index{OllivierRicciCurvature (class in graph\_ricci\_curvature.ollivier\_ricci\_curvature)@\spxentry{OllivierRicciCurvature}\spxextra{class in graph\_ricci\_curvature.ollivier\_ricci\_curvature}}

\begin{fulllineitems}
\phantomsection\label{\detokenize{graph_ricci_curvature:graph_ricci_curvature.ollivier_ricci_curvature.OllivierRicciCurvature}}
\pysigstartsignatures
\pysiglinewithargsret{\sphinxbfcode{\sphinxupquote{class\DUrole{w}{ }}}\sphinxcode{\sphinxupquote{graph\_ricci\_curvature.ollivier\_ricci\_curvature.}}\sphinxbfcode{\sphinxupquote{OllivierRicciCurvature}}}{\sphinxparam{\DUrole{n}{G}\DUrole{p}{:}\DUrole{w}{ }\DUrole{n}{Graph}}\sphinxparamcomma \sphinxparam{\DUrole{n}{edge\_weight\_key}\DUrole{o}{=}\DUrole{default_value}{\textquotesingle{}weight\textquotesingle{}}}\sphinxparamcomma \sphinxparam{\DUrole{n}{node\_weight\_key}\DUrole{o}{=}\DUrole{default_value}{\textquotesingle{}weight\textquotesingle{}}}}{}
\pysigstopsignatures
\sphinxAtStartPar
Bases: \sphinxcode{\sphinxupquote{\_RicciCurvature}}

\sphinxAtStartPar
Class for calculating Ollivier Ricci Curvature of a connected graph. Only
edge weights are considered in Ollivier curvature and are set to 1.0 if values
are not provided in user or found in the input networkx graph object.


\subsubsection{Parameters}
\label{\detokenize{graph_ricci_curvature:id3}}\begin{description}
\sphinxlineitem{G}{[}networkx graph{]}
\sphinxAtStartPar
Input graph

\sphinxlineitem{edge\_weight\_key}{[}str{]}
\sphinxAtStartPar
Key to specify edge weights in networkx graph. Default = weight.

\sphinxlineitem{node\_weight\_key}{[}str{]}
\sphinxAtStartPar
Key to specify node weights in networkx graph. Default = weight.

\end{description}
\index{calculate\_edge\_curvature() (graph\_ricci\_curvature.ollivier\_ricci\_curvature.OllivierRicciCurvature method)@\spxentry{calculate\_edge\_curvature()}\spxextra{graph\_ricci\_curvature.ollivier\_ricci\_curvature.OllivierRicciCurvature method}}

\begin{fulllineitems}
\phantomsection\label{\detokenize{graph_ricci_curvature:graph_ricci_curvature.ollivier_ricci_curvature.OllivierRicciCurvature.calculate_edge_curvature}}
\pysigstartsignatures
\pysiglinewithargsret{\sphinxbfcode{\sphinxupquote{calculate\_edge\_curvature}}}{\sphinxparam{\DUrole{n}{source\_node}}\sphinxparamcomma \sphinxparam{\DUrole{n}{target\_node}}\sphinxparamcomma \sphinxparam{\DUrole{n}{alpha}\DUrole{o}{=}\DUrole{default_value}{0.5}}\sphinxparamcomma \sphinxparam{\DUrole{n}{dist\_type}\DUrole{o}{=}\DUrole{default_value}{\textquotesingle{}uniform\textquotesingle{}}}\sphinxparamcomma \sphinxparam{\DUrole{n}{method}\DUrole{o}{=}\DUrole{default_value}{\textquotesingle{}otd\textquotesingle{}}}\sphinxparamcomma \sphinxparam{\DUrole{n}{weight\_path\_matrix}\DUrole{o}{=}\DUrole{default_value}{False}}\sphinxparamcomma \sphinxparam{\DUrole{n}{numThreads}\DUrole{o}{=}\DUrole{default_value}{1}}\sphinxparamcomma \sphinxparam{\DUrole{n}{reg}\DUrole{o}{=}\DUrole{default_value}{0.1}}}{}
\pysigstopsignatures
\sphinxAtStartPar
Calculate value of Ollivier Ricci Curvature tensor associated with an edge
between a source and target node defined as

\sphinxAtStartPar
1 \sphinxhyphen{} ( Wasserstein 1 Distance / Edge Weight )


\paragraph{Parameters}
\label{\detokenize{graph_ricci_curvature:id4}}\begin{description}
\sphinxlineitem{source\_node}{[}int or tuple{]}
\sphinxAtStartPar
index of source\_node in graph self.G

\sphinxlineitem{target\_node}{[}int or tuple{]}
\sphinxAtStartPar
index of target node in graph self.G

\sphinxlineitem{alpha}{[}float{]}
\sphinxAtStartPar
hyperparameter (0 \textless{}= alpha \textless{}=1) determining how much mass to move
from node

\sphinxlineitem{dist\_type}{[}str{]}
\sphinxAtStartPar
Distribution type for mass distribution in source or target node neighborhood. Default: uniform. Options: uniform, linear, inverse\sphinxhyphen{}linear, gaussian.

\sphinxlineitem{method}{[}str{]}
\sphinxAtStartPar
Method for calculating optimal transport plan. Options: otd (optimal transport distance), sinkhorn

\sphinxlineitem{weight\_path\_matrix}{[}bool{]}
\sphinxAtStartPar
When True, use edge weights when calculating shortest distance matrix. Default: False.

\sphinxlineitem{numThreads}{[}int{]}
\sphinxAtStartPar
Specify number of threads for optimal transport plan. Only for “otd” method.

\sphinxlineitem{reg}{[}float{]}
\sphinxAtStartPar
Regularization term to be used with “sinkhorn” method

\end{description}


\paragraph{Returns}
\label{\detokenize{graph_ricci_curvature:id5}}\begin{description}
\sphinxlineitem{curvature}{[}float{]}
\sphinxAtStartPar
value of curvature tensor

\end{description}

\end{fulllineitems}

\index{calculate\_ricci\_curvature() (graph\_ricci\_curvature.ollivier\_ricci\_curvature.OllivierRicciCurvature method)@\spxentry{calculate\_ricci\_curvature()}\spxextra{graph\_ricci\_curvature.ollivier\_ricci\_curvature.OllivierRicciCurvature method}}

\begin{fulllineitems}
\phantomsection\label{\detokenize{graph_ricci_curvature:graph_ricci_curvature.ollivier_ricci_curvature.OllivierRicciCurvature.calculate_ricci_curvature}}
\pysigstartsignatures
\pysiglinewithargsret{\sphinxbfcode{\sphinxupquote{calculate\_ricci\_curvature}}}{\sphinxparam{\DUrole{n}{alpha}\DUrole{o}{=}\DUrole{default_value}{0.5}}\sphinxparamcomma \sphinxparam{\DUrole{n}{norm}\DUrole{o}{=}\DUrole{default_value}{True}}\sphinxparamcomma \sphinxparam{\DUrole{n}{dist\_type}\DUrole{o}{=}\DUrole{default_value}{\textquotesingle{}uniform\textquotesingle{}}}\sphinxparamcomma \sphinxparam{\DUrole{n}{method}\DUrole{o}{=}\DUrole{default_value}{\textquotesingle{}otd\textquotesingle{}}}\sphinxparamcomma \sphinxparam{\DUrole{n}{weight\_path\_matrix}\DUrole{o}{=}\DUrole{default_value}{False}}\sphinxparamcomma \sphinxparam{\DUrole{n}{numThreads}\DUrole{o}{=}\DUrole{default_value}{1}}\sphinxparamcomma \sphinxparam{\DUrole{n}{reg}\DUrole{o}{=}\DUrole{default_value}{0.1}}}{}
\pysigstopsignatures
\sphinxAtStartPar
Calculate nonzero values of Ricci curvature tensor for all edges in
graph self.G and tensor contractions.


\paragraph{Parameters}
\label{\detokenize{graph_ricci_curvature:id6}}\begin{description}
\sphinxlineitem{alpha}{[}float{]}
\sphinxAtStartPar
Hyperparameter (0 \textless{}= alpha \textless{}=1) determining how much mass to move
from node.

\sphinxlineitem{norm}{[}bool{]}
\sphinxAtStartPar
If True, normalize nodal scalar curvature.

\sphinxlineitem{dist\_type}{[}str{]}
\sphinxAtStartPar
Distribution type for mass distribution in source or target node neighborhood. Default: uniform. Options: uniform, linear, inverse\sphinxhyphen{}linear, gaussian.

\sphinxlineitem{method}{[}str{]}
\sphinxAtStartPar
Method for calculating optimal transport plan. Options: otd (optimal transport distance), sinkhorn.

\sphinxlineitem{weight\_path\_matrix}{[}bool{]}
\sphinxAtStartPar
When True, use edge weights when calculating shortest distance matrix. Default: False.

\sphinxlineitem{numThreads}{[}int{]}
\sphinxAtStartPar
Specify number of threads for optimal transport plan. Only for “otd” method.

\sphinxlineitem{reg}{[}float{]}
\sphinxAtStartPar
Regularization term to be used with “sinkhorn” method.

\end{description}


\paragraph{Returns}
\label{\detokenize{graph_ricci_curvature:id7}}\begin{description}
\sphinxlineitem{self.G}{[}networkx graph{]}
\sphinxAtStartPar
Returns graph with ricci\_curvature as graph, node, and edge attributes

\end{description}

\end{fulllineitems}


\end{fulllineitems}



\subsection{Module contents}
\label{\detokenize{graph_ricci_curvature:module-graph_ricci_curvature}}\label{\detokenize{graph_ricci_curvature:module-contents}}\index{module@\spxentry{module}!graph\_ricci\_curvature@\spxentry{graph\_ricci\_curvature}}\index{graph\_ricci\_curvature@\spxentry{graph\_ricci\_curvature}!module@\spxentry{module}}

\chapter{Indices and tables}
\label{\detokenize{index:indices-and-tables}}\begin{itemize}
\item {} 
\sphinxAtStartPar
\DUrole{xref,std,std-ref}{genindex}

\item {} 
\sphinxAtStartPar
\DUrole{xref,std,std-ref}{modindex}

\item {} 
\sphinxAtStartPar
\DUrole{xref,std,std-ref}{search}

\end{itemize}


\renewcommand{\indexname}{Python Module Index}
\begin{sphinxtheindex}
\let\bigletter\sphinxstyleindexlettergroup
\bigletter{g}
\item\relax\sphinxstyleindexentry{graph\_ricci\_curvature}\sphinxstyleindexpageref{graph_ricci_curvature:\detokenize{module-graph_ricci_curvature}}
\item\relax\sphinxstyleindexentry{graph\_ricci\_curvature.forman\_ricci\_curvature}\sphinxstyleindexpageref{graph_ricci_curvature:\detokenize{module-graph_ricci_curvature.forman_ricci_curvature}}
\item\relax\sphinxstyleindexentry{graph\_ricci\_curvature.ollivier\_ricci\_curvature}\sphinxstyleindexpageref{graph_ricci_curvature:\detokenize{module-graph_ricci_curvature.ollivier_ricci_curvature}}
\end{sphinxtheindex}

\renewcommand{\indexname}{Index}
\printindex
\end{document}